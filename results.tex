This section details our experimental methodology and the  video  datasets used. We evaluate our approach on the two popular datasets
Youtube Action Dataset and the Olympic Sports dataset. On both datasets, we show that our work exceeds the current state-of-the-art
results, by comparing with other popular techniques. Also, we vary the contribution of static and motion features, for the calculation
of combined vector series, and explore what is optimum contribution from each domain. By this step, we also prove that both static and motion features
provide vital information about the actions occuring in a video. Furthemore, we highlight the importance of considering the time evolution
of sub activities, in order to identify a complex event, by comparing the results of RNN and other methodologies, which does not capture
the temporal dynamics. 

Datasets


Comparison with the state-of-the-art

The summery of comparison of our work, is shown in table. $\rho = \frac{1}{\sqrt{2}}$ is used to combine the static and motion vectors.

Youtubedataset
We were able to achieve an accuracy over 90\% for every class. We achieve an overall accuracy of 93.3\%. Our results are compared with the
algorithms, . As it is clearly visible, our method surpasses the other
methods in all the classes except in diving, golf swinging
and volleyball spiking . Overall, our approach exceeded currently best reported results, for this dataset, to the best of our knowledge, 
by a significant 8\% margin.