This paper presents an end to end system for action classification which operates
on both static and motion features. Our approach relies on deep features,
for creating static vectors, and \textit{motion tubes} for motion features.
Motion tubes are a novel concept we introduce in this paper which can be
used to track individual actors or objects across frames, and model micro level actions.
We present three novel methods: Based on Cholesky transformation, variance ratio, and PCA for efficient combining of features
from different domains, which is a vital requirement in action classification.
Cholesky method provide the power to control the contribution of each domain in exact numbers, and
variance ratio based method mathematically provides an optimum ratio for contribution. We show that these mathematical
and experimental values agree with each other. We run experiments to show that the accuracy depends on the ratio of this contribution and the optimum contribution of
static and motion domains may vary depending on the richness of the motion information.

Through our experiments we also show that our static and motion features are complementary,
and contribute to the final result. We also compare our three fusion algorithms, and
show that Choleky based method is superior, although all three of them give impressive results. We also model the temporal progression of sub-events using an LSTM network. Experimental
results indicate that this is indeed beneficiary, compared to using models which does not capture temporal dynamics. Comparison of our work with multiple state-of-the-art algorithms, on the two popular data sets, UCF-11 and Hollywood2, show that our system performs better.


In the future, it would be interesting to improve the motion tubes, so that, it can maintain an identity over each actor object.
While it is mostly the case even in the present system, there is no guarantee. 
Also exploring more powerful methods to describe micro actions inside motion tubes would also be interesting, since it may increase the descriptiveness of the motion features and
contribute well to the final accuracy.
