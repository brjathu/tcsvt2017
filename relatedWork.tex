
There has been not many approaches in activity recognition, which highlight the
importance of exclusively engineering static and motion features. Most of the work
relies on generating spatio-temporal interest regions such as action tubes \cite{gkioxari2015finding},
tubelets \cite{jain2014action}, dense trajectory based methods \cite{van2015apt}, \cite{wang2015action},
spatio-temporal extensions of spatial pyramid pooling \cite{laptev2008learning},
or spatioi-temporal low level features  \cite{schuldt2004recognizing}, \cite{ke2005efficient}, \cite{shechtman2005space},
\cite{wang2011action}, \cite{klaser2008spatio}, \cite{yu2010real}. Action tubes \cite{gkioxari2015finding} is quite similar to 
our motion tubes, but our motion candidate regions are chosen based on more powerful dense trajectories \cite{wang2011action} instead of
raw optic flows. Also, we employ a tracking mechanism of each moving area through motion tubes isolating actions 
of each actor throughout the video. Our static interest regions are independent from motion
unlike in Gkioxari \textit{et al.}\cite{gkioxari2015finding}, where we are able to extract background scenery information using CNNS, for 
action recognition. 
One of the common attributes of these methods is that motion density is the 
dominant factor for identifying candidate regions. 
In contrast, in this study we treat motion and static features 
as two independent domains and eleminate dominance factor. 

A few attempts has recently been made on exclusive crafting and late fusion 
of motion and static features. Simonyan \textit{et al.}\cite{simonyan2014two} first decomposes a video in to 
spatial and temporal components based on RGB and optical flow frames.
Then they aply two deep CNNs on these two components separately to extract spacial and 
temporal information. Each network operates mutually exclusively and performs action classification
independently. Afterwards, softmax scores are coalesced by late fusion. 

Work done in Feichtenhofer \textit{et al.}\cite{feichtenhofer2016convolutional} is also similar. Instead of late fusion, 
they fuse the two domains in a convolutional layer. Both these approaches rely explicitly on automatic feature 
generation in increasingly abstract layers. While this has provided promising results on static feature generation,
we argue that motion patterns can be more more superiorly extracted by hand crafted features. This is because
temporal dynamics extend to a longer motion duration unlike spatial variations. It is not possible 
to capture and discriminate motion patterns in to classes by a system which has a smaller temporal support. There are models
which employ 3D convolution \cite{ji20133d}, \cite{tran2015learning}, which extends the traditional CNNs into temporal domain. 
Sameera \textit{et al.}\cite{7486474} applies CNNs on optic flows, and Kim \textit{et al.}\cite{kim2007human} on low level hand-crafted inputs 
(spatio-temporal outer boundaries volumes), to extract motion features. However, even by generating hierarchical 
features on top of pixel level features, it is not easy to discriminate motion classes as it does not extend to a long range. 
Also, tracking and modeling actions of each actor separately in longer time durations is not possible with these 
approaches. Our motion features, on the other hand, are capable of capturing motion patterns in longer temporal durations.
Furthermore, with the aid of \textit{motion tubes} our system tracks and models the activities of each moving area separately. 

Regarding video evolution, Fernando \textit{et al.}\cite{fernando2015modeling} postulate
a function capable of ordering the frames of a video
temporally. They learn a ranking function per video using a ranking machine and use the learned parameters as 
a video descriptor. Several other methodologies, e.g., HMM (\cite{wang2011hidden}, \cite{wu2014leveraging}), 
CRF-based methods (\cite{song2013action}), also have been employed in this aspect. These methods model the video evolution in frame 
level. In contrast, attempts for temporal ordering of atomic events also has been carried out \cite{rohrbach2012script}, \cite{bhattacharya2014recognition}.
In Rohrbach \textit{et al.}\cite{rohrbach2012script}, transition probabilities of a series of events are encoded statistically with a HMM model. 
In Bhattacharya \textit{et al.}\cite{bhattacharya2014recognition}, they identify low level actions using dense trajectories and assign concept identity 
probabilities for each action. They apply a LDS on these generated concept vectors to exploit temporal 
dynamics of the low level actions. Li \textit{et al.}\cite{li2013recognizing} uses simple dynamical systems \cite{jackson1992perspectives},
\cite{kailath1974view} to create a dictionary of low-level spatio-temporal attributes and these attributes 
are later used as a histogram to represent high level actions. Our method too follows a similar approach,
as we also generate descriptors for sub events and then extract temporal progression 
of these sub events. However, instead of simple statistical models, which has a finite dynamic response, 
we use a LSTM network \cite{hochreiter1997long} to capture these dynamics. In action recognition literature,
such models are starting appear. In Yue-Hei \textit{et al.}
\cite{yue2015beyond} the LSTM network model the dynamics of the CNN activations, and in Donahue \textit{et al.}\cite{donahue2015long},
the LSTM network learns the temporal dynamics of low level features, generated by a CNN. 

